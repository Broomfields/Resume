%%%%%%%%%%%%%%%%%%%%%%%%%%%%%%%%%%%%%%%%%
% Developer CV
% LaTeX Template
% Version 1.0 (28/1/19)
%
% This template originates from:
% http://www.LaTeXTemplates.com
%
% Authors:
% Jan Vorisek (jan@vorisek.me)
% Based on a template by Jan Küster (info@jankuester.com)
% Modified for LaTeX Templates by Vel (vel@LaTeXTemplates.com)
%
% License:
% The MIT License (see included LICENSE file)
%
%%%%%%%%%%%%%%%%%%%%%%%%%%%%%%%%%%%%%%%%%

%----------------------------------------------------------------------------------------
%	PACKAGES AND OTHER DOCUMENT CONFIGURATIONS
%----------------------------------------------------------------------------------------

\documentclass[9pt]{developercv} % Default font size, values from 8-12pt are recommended

%----------------------------------------------------------------------------------------

\begin{document}

%----------------------------------------------------------------------------------------
%	TITLE AND CONTACT INFORMATION
%----------------------------------------------------------------------------------------

\begin{minipage}[t]{0.45\textwidth} % 45% of the page width for name
	\vspace{-\baselineskip} % Required for vertically aligning minipages
	
	% If your name is very short, use just one of the lines below
	% If your name is very long, reduce the font size or make the minipage wider and reduce the others proportionately
	\colorbox{black}{{\HUGE\textcolor{white}{\textbf{\MakeUppercase{Shaun}}}}} % First name
	
	\colorbox{black}{{\HUGE\textcolor{white}{\textbf{\MakeUppercase{Broomfield}}}}} % Last name
	
	\vspace{6pt}
	
	{\huge Software Engineer} % Career or current job title
\end{minipage}
\begin{minipage}[t]{0.33\textwidth} % 27.5% of the page width for the first row of icons
	\vspace{-\baselineskip} % Required for vertically aligning minipages
	
	% The first parameter is the FontAwesome icon name, the second is the box size and the third is the text
	% Other icons can be found by referring to fontawesome.pdf (supplied with the template) and using the word after \fa in the command for the icon you want
	\icon{MapMarker}{12}{Ingleby Barwick, England}\\
	
	\vspace{0.05cm}
	
	\icon{At}{12}{\href{mailto:contact@shaunbroomfield.com}{contact@shaunbroomfield.com}}\\
	% \icon{Phone}{12}{+44 7714825052}\\	
\end{minipage}
\begin{minipage}[t]{0.24\textwidth} % 27.5% of the page width for the second row of icons
	\vspace{-\baselineskip} % Required for vertically aligning minipages
	
	% The first parameter is the FontAwesome icon name, the second is the box size and the third is the text
	% Other icons can be found by referring to fontawesome.pdf (supplied with the template) and using the word after \fa in the command for the icon you want
	\icon{Calendar}{12}{Rev: 2024/10/13}\\
	
	\vspace{0.05cm}
	
	% \icon{Globe}{12}{\href{https://shaunbroomfield.com/}{ShaunBroomfield.com}}\\
	
	\icon{Linkedin}{12}{\href{https://www.linkedin.com/in/shaunbroomfield/}{ShaunBroomfield}}\\
	
	% \icon{Github}{12}{\href{https://github.com/Broomfields}{Broomfields}}\\
\end{minipage}

\vspace{0.5cm}

%----------------------------------------------------------------------------------------
%	INTRODUCTION
%----------------------------------------------------------------------------------------

\begin{minipage}[t]{1\textwidth} % 40% of the page width for the introduction text

	\cvsect{Who Am I?}

	I’m a full-stack Software Engineer with a diverse skill set, rooted in C, C++, Microsoft SQL Server, MFC, and Windows Forms development for the Windows platform. My extensive experience in crafting SCADA software has involved building complex systems with microservice-like architectures, comprised of numerous intertwined programs, drivers, and services that seamlessly synchronise across Windows environments.

	\vspace{0.5cm}

	Over the past year, I have expanded my technical repertoire by focusing on modern .NET technologies, specifically C\#, .NET 8, ASP.NET Core, Entity Framework, and GIT. I’ve led the development of REST APIs and Blazor web applications, utilising these tools to build dynamic, responsive solutions while maintaining my expertise in traditional Windows development.

	\vspace{0.5cm}
	
	This unique blend of skills allows me to bridge the gap between legacy and modern software development, ensuring that I can deliver solutions that leverage the best of both worlds—whether it’s enhancing SCADA systems with C++ or building scalable web applications with C\# and .NET 8.

	\vspace{0.5cm}

	Over the years, I’ve had many hobbies and interests, but recently I’ve found joy in a few creative and relaxing pastimes. I love making pizza, using my pizza oven to experiment with different recipes and techniques. I’m also infatuated with books, where I lose myself in different worlds, exploring a wide range of genres and stories.

\end{minipage}

\vspace{1cm}	


%----------------------------------------------------------------------------------------
%	Career
%----------------------------------------------------------------------------------------

\cvsect{Career}

\begin{entrylist}
	\entry
		{1/2020 -- Present\\4 Years, 10 Months}
		{Software Engineer}
		{Tascomp Limited}
		{Billingham, England, United Kingdom
		
		\begin{itemize}
		    \item[\ding{117}] Led improvements to the company's MFC-Dialogue-based IDE for 'SLANG,' incorporating vital features such as undo/redo and source control.
		    \item[\ding{117}] Managed end-to-end processes for key SCADA interface applications and internal tools, from conception to documentation.
		    \item[\ding{117}] Engineered bespoke full-stack applications, using MFC-based C++ and .NET Framework's Windows Forms, tailored for compatibility within the company’s micro-service architecture.
		    \item[\ding{117}] Crafted specialised MFC-based C++ drivers to enhance large SCADA systems.
		    \item[\ding{117}] Conducted comprehensive pre-FAT tests for airport SCADA systems, contributing to essential documentation.
		    \item[\ding{117}] Extended the 'SLANG' language by developing multiple 'SLANG Extensions' through MFC-based C++.
		\end{itemize}

		}
	\entry
		{6/2019 -- 1/2020\\8 Months}
		{Apprentice Software Engineer}
		{}
		{
		\begin{itemize}
		    \item[\ding{117}] Enhanced system security by updating and replacing a crucial 'C' file decryption library.
		    \item[\ding{117}] Developed innovative SCADA solutions through procedural finite-state-machine strategies using the 'SLANG' domain language.
		    \item[\ding{117}] Created tailored MFC-based C++ solutions for the Windows platform, optimised for compatibility with our micro-service architecture.
		    \item[\ding{117}] Assembled and configured Windows PC systems to meet varied customer requirements.
		\end{itemize}
		
		\vspace{1em}
		\icon{Globe}{12}{\href{https://www.tascomp.com/}{tascomp.com}}
		\hspace{2em}
		\icon{Linkedin}{12}{\href{https://www.linkedin.com/company/tascomp/}{Tascomp Limited}}
		\vspace{1em}
		}
	\entry
		{7/2018 -- 1/2019\\7 Months}
		{Apprentice Software Engineer}
		{Labman Automation}
		{Seamer, England, United Kingdom
		
		\begin{itemize}
		    \item[\ding{117}] Contributed to improving internal IT operations for enhanced efficiency.
		    \item[\ding{117}] Played a key role in the development of a unique medicinal fabrication system:
		    \begin{itemize}
		        \item[\ding{118}] Applied knowledge of Beckhoff IPC Programming in Structured Text and Ladder Logic, ensuring seamless communication with a spectrometer via Modbus TCP/IP.
		        \item[\ding{118}] Designed MySQL Stored Procedures optimised for concurrent C++ development in the project.
		    \end{itemize}
		\end{itemize}

		\vspace{1em}
		\icon{Globe}{12}{\href{https://www.labmanautomation.com/}{labmanautomation.com}}
		\hspace{2em}
		\icon{Linkedin}{12}{\href{https://www.linkedin.com/company/labman-automation-ltd/}{Labman-Automation-Ltd}}
		\vspace{1em}
		}
		

\end{entrylist}

\vspace{0.5cm}	


%----------------------------------------------------------------------------------------
%	EDUCATION
%----------------------------------------------------------------------------------------

\cvsect{Education}

\begin{entrylist}
	\entry
		{12/2018 -- 01/2020}
		{C\&G Level 3 Diploma in ICT Professional Competence}
		{Middlesbrough College}
		{Computer Software Engineering}
	\entry
		{7/2016}
		{Royal Institution Engineering Masterclasses}
		{The Royal Institution}
		{}
\end{entrylist}

\cvsect{Certificates}
\begin{entrylist}
	\entry
		{12/2019}
		{Certificate of Unit Credit Towards Level 3 in ICT Systems and Principles}
		{City \& Guilds}
		{}
	\entry
		{12/2019}
		{Level 3 Certificate in ICT Systems and Principles}
		{City \& Guilds}
		{}
	\entry
		{12/2019}
		{Certificate of Unit Credit Towards Level 3 Diploma in ICT Professional Competence}
		{City \& Guilds}
		{}
	\entry
		{12/2019}
		{Level 3 Diploma in ICT Professional Competence}
		{City \& Guilds}
		{}
\end{entrylist}

\vspace{0.5cm}	

%----------------------------------------------------------------------------------------

\end{document}
