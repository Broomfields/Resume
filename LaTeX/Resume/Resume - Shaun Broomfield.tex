%%%%%%%%%%%%%%%%%%%%%%%%%%%%%%%%%%%%%%%%%
% Developer CV
% LaTeX Template
% Version 1.0 (28/1/19)
%
% This template originates from:
% http://www.LaTeXTemplates.com
%
% Authors:
% Jan Vorisek (jan@vorisek.me)
% Based on a template by Jan Küster (info@jankuester.com)
% Modified for LaTeX Templates by Vel (vel@LaTeXTemplates.com)
%
% License:
% The MIT License (see included LICENSE file)
%
%%%%%%%%%%%%%%%%%%%%%%%%%%%%%%%%%%%%%%%%%

%----------------------------------------------------------------------------------------
%	PACKAGES AND OTHER DOCUMENT CONFIGURATIONS
%----------------------------------------------------------------------------------------

\documentclass[9pt]{developercv} % Default font size, values from 8-12pt are recommended

%----------------------------------------------------------------------------------------

\begin{document}

%----------------------------------------------------------------------------------------
%	TITLE AND CONTACT INFORMATION
%----------------------------------------------------------------------------------------

\begin{minipage}[t]{0.45\textwidth} % 45% of the page width for name
	\vspace{-\baselineskip} % Required for vertically aligning minipages
	
	% If your name is very short, use just one of the lines below
	% If your name is very long, reduce the font size or make the minipage wider and reduce the others proportionately
	\colorbox{black}{{\HUGE\textcolor{white}{\textbf{\MakeUppercase{Shaun}}}}} % First name
	
	\colorbox{black}{{\HUGE\textcolor{white}{\textbf{\MakeUppercase{Broomfield}}}}} % Last name
	
	\vspace{6pt}
	
	{\huge Software Engineer} % Career or current job title
\end{minipage}
\begin{minipage}[t]{0.33\textwidth} % 27.5% of the page width for the first row of icons
	\vspace{-\baselineskip} % Required for vertically aligning minipages
	
	% The first parameter is the FontAwesome icon name, the second is the box size and the third is the text
	% Other icons can be found by referring to fontawesome.pdf (supplied with the template) and using the word after \fa in the command for the icon you want
	\icon{MapMarker}{12}{Hartlepool, England}\\
	\icon{At}{12}{\href{mailto:contact@shaunbroomfield.com}{contact@shaunbroomfield.com}}\\
	\icon{Phone}{12}{+44 7714825052}\\	
\end{minipage}
\begin{minipage}[t]{0.24\textwidth} % 27.5% of the page width for the second row of icons
	\vspace{-\baselineskip} % Required for vertically aligning minipages
	
	% The first parameter is the FontAwesome icon name, the second is the box size and the third is the text
	% Other icons can be found by referring to fontawesome.pdf (supplied with the template) and using the word after \fa in the command for the icon you want
	\icon{Globe}{12}{\href{https://shaunbroomfield.com/}{ShaunBroomfield.com}}\\
	\icon{Linkedin}{12}{\href{https://www.linkedin.com/in/shaunbroomfield/}{ShaunBroomfield}}\\
	\icon{Github}{12}{\href{https://github.com/Broomfields}{Broomfields}}\\
\end{minipage}

\vspace{0.5cm}

%----------------------------------------------------------------------------------------
%	INTRODUCTION
%----------------------------------------------------------------------------------------

\begin{minipage}[t]{1\textwidth} % 40% of the page width for the introduction text

	\cvsect{Who Am I?}

	I'm a full-stack Software Engineer with a predominant background in developing SCADA software for the Windows operating system. My deep-rooted expertise is within the Microsoft ecosystem, leveraging tools like C, C++, C++/CLI, SQL Server, MFC, and .NET Framework.

	\vspace{0.5cm}	

	Currently, my professional role involves crafting a SCADA software product that embodies a Micro-Services-like architecture on the Windows platform. This sophisticated system comprises hundreds of intertwined programs, drivers, and services that synchronise seamlessly. Their collaboration relies on data-modules, queues, stacks, pools, and an integration of both relational and in-house non-relational databases, ultimately furnishing a comprehensive SCADA solution across one or multiple Windows machines.
	
	\vspace{0.5cm}	

	Outside of my professional realm, I possess a broad curiosity for various programming fields. While I have some experience tinkering in web and app development, my primary personal pursuits revolve around systems-level languages, libraries, and frameworks. This has led me to explore the intricacies of graphics programming, physics programming, embedded systems, and other related computing arenas.

\end{minipage}

\vspace{1cm}	

%----------------------------------------------------------------------------------------
%	SKILLS AND TECHNOLOGIES
%----------------------------------------------------------------------------------------

\begin{minipage}[t]{0.5\textwidth} % 50% of the page for the skills bar chart

	\cvsect{Programming Language Experience}

	\begin{barchart}{5.5}
		\baritem{C}{60}
		\baritem{C++}{80}
		\baritem{C++/CLI}{100}
		\baritem{C\#}{40}
		\baritem{Lua}{35}
		\baritem{Java}{30}
		\baritem{SLANG}{80}
		\baritem{Git}{30}
		\baritem{Markdown}{40}
		\baritem{LaTeX}{20}
	\end{barchart}

\end{minipage}
\hfill % Whitespace between
\begin{minipage}[t]{0.6\textwidth} % 40% of the page width for the introduction text
	
	\cvsect{Platform Development Experience}

	% Adapted from @Jake's answer from http://tex.stackexchange.com/a/82729/226
	% \wheelchart{outer radius}{inner radius}{
	% comma-separated list of value/text width/color/detail}
	\wheelchart{1.5cm}{0.5cm}{%
		6/8em/black!25/Windows Development,
		3/8em/black!55/Win32,
		4.5/10em/black!80/MFC,
		6/8em/black!45/.NET Framework,
		6/10em/black!65/Windows Forms
	}

\end{minipage}

\vspace{1cm}	

%----------------------------------------------------------------------------------------
%	Career
%----------------------------------------------------------------------------------------

\cvsect{Career}

\begin{entrylist}
	\entry
		{1/2020 -- Present\\3 Years 8 Months}
		{Software Engineer}
		{Tascomp}
		{Billingham, England, United Kingdom\\
		
		\lorem \lorem \lorem \lorem \lorem % Dummy text
		
		\vspace{0.5em}	

		\icon{Globe}{12}{\href{https://www.tascomp.com/}{tascomp.com}}
		\hspace{2em}
		\icon{Linkedin}{12}{\href{https://www.linkedin.com/company/tascomp/}{Tascomp}}}\\
	\entry
		{6/2019 -- 1/2020\\8 Months}
		{Apprentice Software Engineer}
		{Tascomp}
		{Billingham, England, United Kingdom\\
		
		\lorem \lorem \lorem \lorem \lorem % Dummy text
		
		\vspace{0.5em}	

		\icon{Globe}{12}{\href{https://www.tascomp.com/}{tascomp.com}}
		\hspace{2em}
		\icon{Linkedin}{12}{\href{https://www.linkedin.com/company/tascomp/}{Tascomp}}}\\
	\entry
		{2/2019 -- 4/2019\\2 Months}
		{Apprentice Web Developer}
		{Admit Me}
		{Middlesbrough, England, United Kingdom\\ 
		
		During this apprenticeship, I expanded my knowledge beyond basic HTML, CSS, and JS by immersing myself in a range of new platforms, concepts, languages, and frameworks. While this role offered invaluable insights, the tenure was brief due to the start-up's instability, which proved challenging for apprenticeship growth.
		}\\
	\entry
		{7/2018 -- 1/2019\\7 Months}
		{Apprentice Software Engineer}
		{Labman Automation}
		{Seamer, England, United Kingdom\\ 
		
		About a week after finishing my GCSEs, 16 years old and fresh out of school, I started my career at a robotics company and was ready for whatever the world could throw at me. Since working at Labman was my first ever "real" job, I wasn't quite 
		
		\vspace{0.5em}	

		\icon{Globe}{12}{\href{https://www.labmanautomation.com/}{labmanautomation.com}}
		\hspace{2em}
		\icon{Linkedin}{12}{\href{https://www.linkedin.com/company/labman-automation-ltd/}{Labman-Automation-Ltd}}}\\
\end{entrylist}

\vspace{1cm}	

%----------------------------------------------------------------------------------------
%	EDUCATION
%----------------------------------------------------------------------------------------

\cvsect{Education}

\begin{entrylist}
	\entry
		{9/2022 -- 9/2025}
		{Bachelor's Degree in Computer Software Engineering (In Progress)}
		{Teesside University}
		{}
	\entry
		{12/2018 -- 12/2020}
		{C\&G Level 3 Diploma in ICT Professional Competence}
		{Middlesbrough College}
		{Computer Software Engineering}
	\entry
		{7/2016}
		{Royal Institution Engineering Masterclasses}
		{The Royal Institution}
		{}
\end{entrylist}

\cvsect{Certificates}
\begin{entrylist}
	\entry
		{12/2019}
		{Certificate of Unit Credit Towards Level 3 in ICT Systems and Principles}
		{City \& Guilds}
		{}
	\entry
		{12/2019}
		{Level 3 Certificate in ICT Systems and Principles}
		{City \& Guilds}
		{}
	\entry
		{12/2019}
		{Certificate of Unit Credit Towards Level 3 Diploma in ICT Professional Competence}
		{City \& Guilds}
		{}
	\entry
		{12/2019}
		{Level 3 Diploma in ICT Professional Competence}
		{City \& Guilds}
		{}
	\entry
		{12/2019}
		{Level 3 Diploma in ICT Professional Competence}
		{City \& Guilds}
		{}
\end{entrylist}

\vspace{1cm}	

%----------------------------------------------------------------------------------------
%	ADDITIONAL INFORMATION
%----------------------------------------------------------------------------------------

\begin{minipage}[t]{0.3\textwidth}
	\vspace{-\baselineskip} % Required for vertically aligning minipages

	\cvsect{Languages}
	
	\textbf{English} - native\\


	\cvsect{Forms of Work}
	
	\Label{black}{Office} \Label{black}{Hybrid} \Label{black}{Remote}\\

	\begin{minipage}[t]{0.8\textwidth}
		When it comes to where I work, I don't mind working in an office, working from home, or a bit of both.\\ 

		As long as my commute doesn't take 2 hours,  I'm happy to travel for work!
	\end{minipage}
\end{minipage}
\hfill
\begin{minipage}[t]{0.7\textwidth}
	\vspace{-\baselineskip} % Required for vertically aligning minipages
	
	\cvsect{Hobbies}
	
	At the heart of my hobbies lies a theme: creation. 
	It's the common thread that ties together my diverse interests. \\

	In the world of 3D Printing and Modelling using SCAD, I enjoy transforming a digital design into something tangible, be it a functional item, prop, or detailed model. \\

	Similarly, while programming, I find joy in crafting lines of programming into functional solutions, bringing digital creation to life and often interacting with the real word in some shape or form. \\

	This drive to create even extends to my kitchen adventures. Recently, cooking has become another avenue for me to experiment and express, using ingredients to explore delicious expressions of different cultures.\\

	When I need a moment's pause from creating, I take refuge in the worlds of Sci-Fi, Fantasy, and Mythology through audio books. They offer an escape and often reignite my creative spark.


\end{minipage}

\vspace{1cm}	

%----------------------------------------------------------------------------------------
%	Programming Journey
%----------------------------------------------------------------------------------------

\cvsect{Programming Journey}

\begin{entrylist}
	\entry
		{Primary School}
		{Lua with ComputerCraft}
		{Personal}
		{\\
			My initiation into the world of programming began in 2012, thanks to Minecraft. I delved into modded Minecraft and stumbled upon ComputerCraft, a mod that ingeniously incorporated DOS-like computers within the game. Fascinatingly, these virtual computers were programmable using Lua, enabling me to tinker with existing programs and eventually create and execute my first programs.

			\vspace{0.5cm}
			\Label{black}{Lua}
		}
	\entry
		{Secondary School}
		{Exploring core programming concepts through different languages}
		{Personal and Education}
		{\\
			Early in my secondary school days, I was introduced to a few graphical programming languages. These primarily consisted of drag-and-drop interfaces designed for script construction. However, the real intrigue came a couple of years later, when I encountered Visual Basic. Here, I embarked on creating console-based conversational trees, deepening my understanding of logical flows and interactivity.

			\vspace{0.5cm}

			My enthusiasm for Minecraft did not wane; in fact, it spurred a personal challenge. Determined to put my own spin on the game, I delved into Java, teaching myself through a plethora of online tutorials. My perseverance bore fruit as I successfully managed to both create new Minecraft mods and recreate existing ones, allowing me to tailor my gameplay experience and furthering my programming prowess.

			\vspace{0.5cm}

			My GCSE years in secondary school marked another milestone in my programming journey. I was introduced to Python 2.7 in my computer science class, which further refined my programming skills and expanded my knowledge of software development principles.

			\vspace{0.5cm}


			\Label{black}{Java} \Label{black}{Visual Basic} \Label{black}{Python 2.7}
		}
	\entry
		{College}
		{C\# and Beyond in Apprenticeship}
		{Education}
		{\\
			Upon completing secondary school, I wasted no time and immediately began my first apprenticeship. The academic facet of this apprenticeship acquainted me with C\# among other languages, enriching my repertoire and preparing me for the myriad challenges and innovations in the tech industry.

			\vspace{0.5cm}
			\Label{black}{C\#} \Label{black}{SQL} \Label{black}{HTML} \Label{black}{CSS} \Label{black}{JS}
		}
	\entry
		{Labman}
		{<Insert Header>}
		{Work}
		{\\
			<Insert Description>

			\vspace{0.5cm}
			\Label{black}{C\#} \Label{black}{SQL}
		}
	\entry
		{AdmitMe}
		{<Insert Header>}
		{Work}
		{\\
			<Insert Description>

			\vspace{0.5cm}
			\Label{black}{Ruby On Rails} \Label{black}{HTML} \Label{black}{SCSS}
		}
	\entry
		{University}
		{<Insert Header>}
		{Education}
		{\\
			<Insert Description>

			\vspace{0.5cm}
			\Label{black}{Java} \Label{black}{Git} \Label{black}{Markdown} \Label{black}{LaTeX}
		}
	\entry
		{Personal Projects}
		{<Insert Header>}
		{Personal}
		{\\
			<Insert Description>

			\vspace{0.5cm}
			\Label{black}{Lua} \Label{black}{Java} \Label{black}{Dart} \Label{black}{Flutter} \Label{black}{Rust} \Label{black}{SCAD} \Label{black}{C} \Label{black}{C++} \Label{black}{Bash} \Label{black}{Linux}
		}
	\entry
		{Tascomp}
		{<Insert Header>}
		{Work}
		{\\
			<Insert Description>

			\vspace{0.5cm}
			\Label{black}{C} \Label{black}{C++} \Label{black}{C++/CLI} \Label{black}{MFC} \Label{black}{.NET} \Label{black}{SLANG} \Label{black}{SQL Server} \Label{black}{Win32}
		}
\end{entrylist}

\vspace{1cm}	

%----------------------------------------------------------------------------------------
%	References
%----------------------------------------------------------------------------------------

\cvsect{References}

\begin{entrylist}
	\entry
		{Ashley Tizard}
		{Managing Director}
		{Tascomp Limited}
		{
			\vspace{0.5em}	
			\icon{Linkedin}{12}{\href{https://www.linkedin.com/in/ashley-tizard}{Ashley-Tizard}}\\
		}
	\entry
		{Steven Tilly}
		{Lead Software Development Engineer}
		{Tascomp Limited}
		{
			\vspace{0.5em}	
			\icon{Linkedin}{12}{\href{https://www.linkedin.com/in/steven-tilly-8282141a/}{Steven-Tilly-8282141a}}\\
		}
	\entry
		{George Priestner}
		{Senior Software Engineer}
		{Eaton}
		{
			\vspace{0.5em}	
			\icon{Linkedin}{12}{\href{https://www.linkedin.com/in/george-p-5aa9385a/}{George-P-5aa9385a}}\\
		}
	\entry
		{Rhys Maddren}
		{Graduate Developer}
		{Sage}
		{
			\vspace{0.5em}	
			\icon{Linkedin}{12}{\href{https://www.linkedin.com/in/rhys-maddren-133a7b151/}{Rhys-Maddren-133a7b151/}}\\
		}
\end{entrylist}

%----------------------------------------------------------------------------------------

\end{document}
